%% If you need to pass whatever options to xcolor
\PassOptionsToPackage{dvipsnames}{xcolor}

%% If you are using \orcid or academicons
%% icons, make sure you have the academicons
%% option here, and compile with XeLaTeX
%% or LuaLaTeX.
% \documentclass[10pt,a4paper,academicons]{altacv}

%% Use the "normalphoto" option if you want a normal photo instead of cropped to a circle
% \documentclass[10pt,a4paper,normalphoto]{altacv}

\documentclass[10pt,a4paper,ragged2e,withhyper]{altacv}
\newif\ifen
\newif\ifpl
\entrue

\newcommand{\en}[1]{\ifen#1\fi}
\newcommand{\pl}[1]{\ifpl#1\fi}
%% AltaCV uses the fontawesome5 and academicons fonts
%% and packages.
%% See http://texdoc.net/pkg/fontawesome5 and http://texdoc.net/pkg/academicons for full list of symbols. You MUST compile with XeLaTeX or LuaLaTeX if you want to use academicons.

% Change the page layout if you need to
\geometry{left=1.2cm,right=1.2cm,top=1cm,bottom=1cm,columnsep=0.75cm}

% The paracol package lets you typeset columns of text in parallel
\usepackage{paracol}
\usepackage{xcolor}
% Change the font if you want to, depending on whether
% you're using pdflatex or xelatex/lualatex
\ifxetexorluatex
  % If using xelatex or lualatex:
  \setmainfont{Roboto Slab}
  \setsansfont{Lato}
  \renewcommand{\familydefault}{\sfdefault}
\else
  % If using pdflatex:
  \usepackage[rm]{roboto}
  \usepackage[defaultsans]{lato}
  % \usepackage{sourcesanspro}
  \renewcommand{\familydefault}{\sfdefault}
\fi

% ----- LIGHT MODE -----
\definecolor{SlateGrey}{HTML}{2E2E2E}
\definecolor{LightGrey}{HTML}{666666}
\definecolor{PrimaryColor}{HTML}{001F5A}
\definecolor{SecondaryColor}{HTML}{0039AC}
\definecolor{ThirdColor}{HTML}{F3890B}
\definecolor{BackgroundColor}{HTML}{E2E2E2}
\colorlet{name}{PrimaryColor}
\colorlet{tagline}{PrimaryColor}
\colorlet{heading}{PrimaryColor}
\colorlet{headingrule}{ThirdColor}
\colorlet{subheading}{SecondaryColor}
\colorlet{accent}{SecondaryColor}
\colorlet{emphasis}{SlateGrey}
\colorlet{body}{LightGrey}
\pagecolor{BackgroundColor}   
% ----- DARK MODE -----
%\definecolor{BackgroundColor}{HTML}{242424}
%\definecolor{SlateGrey}{HTML}{6F6F6F}
%\definecolor{LightGrey}{HTML}{ABABAB}
%\definecolor{PrimaryColor}{HTML}{3F7FFF}
%\colorlet{name}{PrimaryColor}
%\colorlet{tagline}{PrimaryColor}
%\colorlet{heading}{PrimaryColor}
%\colorlet{headingrule}{PrimaryColor}
%\colorlet{subheading}{PrimaryColor}
%\colorlet{accent}{PrimaryColor}
%\colorlet{emphasis}{LightGrey}
%\colorlet{body}{LightGrey}
%\pagecolor{BackgroundColor}

% Change some fonts, if necessary
\renewcommand{\namefont}{\Huge\rmfamily\bfseries}
\renewcommand{\personalinfofont}{\small\bfseries}
\renewcommand{\cvsectionfont}{\LARGE\rmfamily\bfseries}
\renewcommand{\cvsubsectionfont}{\large\bfseries}

% Change the bullets for itemize and rating marker
% for \cvskill if you want to
\renewcommand{\itemmarker}{{\small\textbullet}}
\renewcommand{\ratingmarker}{\faCircle}

%% sample.bib contains your publications
%% \addbibresource{sample.bib}

\begin{document}
\name{Emil Redzik}
\tagline{}
%% You can add multiple photos on the left or right
\photoL{4cm}{emil-redzik}

\personalinfo{
    \email{emilredzik1994@gmail.com}\smallskip
    \phone{+48-500-353-785}
    \location{
        \pl{Warszawa, Polska}
        \en{Warsaw, Poland}}\\
    \linkedin{emil-redzik}
    \github{eredzik}
}

\makecvheader

%% Depending on your tastes, you may want to make fonts of itemize environments slightly smaller
% \AtBeginEnvironment{itemize}{\small}


%% Set the left/right column width ratio to 6:4.
\columnratio{0}

\begin{paracol}{1}

    % ----- ABOUT ME -----
    \cvsection{
        \en{About Me}
        \pl{O mnie}
    }
    \begin{quote}
        \pl{Data scientist i inżynier danych. Fascynat bayesowskich metod modelowania danych.}
        \en{Data scientist and data engineer fascinated with bayesian methods of data analysis.}
    \end{quote}
    % ----- ABOUT ME -----

    % ----- EXPERIENCE -----
    \cvsection{
        \pl{Doświadczenie}
        \en{Experience}
    }
    \cvevent{
        \pl{Inżynier danych}
        \en{Data engineer}
    }{| Accenture |
        \pl{Branża ubezpieczeniowa}
        \en{Insurance industry firm}
    }{06/2021 -- }{}

    \begin{itemize}
        \item
              \pl{Budowa systemu przetwarzania, integracji i prezentacji danych środowiskowych
                  z wielu różnorodnych źródeł zewnętrznych.}
              \en{Development of data processing, integration and presentation of environment
                  responsibility data from various outside vendors.}

    \end{itemize}
    \textcolor{accent}{
        Spark, Python, GIT, Palantir Foundry
    }

    \divider
    \cvevent{
        \pl{Inżynier danych}
        \en{Data engineer}
    }{| mBank |
        \pl{Departament Hurtowni Danych}
        \en{Department of Data Warehousing}
    }{06/2019 -- 05/2021 }{}

    \begin{itemize}
        \item
              \pl{Rozwój i utrzymanie systemów liczących bieżące ryzyko płynności i raportujących je.}
              \en{Development and maintenance of bank liquidity calculation systems}
        \item
              \pl{Optymalizacje procesu skutkujące skróceniem dziennego czasu oczekiwania analityków na dane do raportowania o ok 50\% (8h -> 4h)}
              \en{Optimized risk calculation process which resulted in 50\% (8h -> 4h) reduction of time in which analysts had to wait for daily data}
        \item
              \pl{Budowa nowych modułów na potrzeby raportowania zarządczego i nadzorczego}
              \en{Development of new modules for management and supervisory reporting}
        \item
              \pl{Budowa pakietu w pythonie do wygodniejszego zarządzania systemem zbudowanym w SAS}
              \en{Created python library for easier use of system for analysts not familiar with SAS}
        \item
              \pl{Migracja systemu liczenia płynności z SAS na pyspark}
              \en{Migration of liquidity calculation system from SAS to pyspark}
    \end{itemize}
    \textcolor{accent}{
        Python, GIT, SAS, Spark, PL/SQL
    }

    \divider
    \cvevent{
        \pl{Specjalista ds. Ryzyka Kredytowego}
        \en{Credit risk specialist}
    }{| PKO S.A. |
        \pl{Departament Ryzyka Kredytowego}
        \en{Credit Risk Department}
    }{02/2018 -- 05/2019}{}
    \begin{itemize}
        \item
              \pl{Przebudowa procesu liczenia bazy klientów pre-approved skutkująca zwiększeniem
                  sprzedaży kredytów przy jednoczesnym braku podwyższenia poziomu ryzyka tej populacji}
              \en{Recreated process of calculating pre-approved customers database resulting in increased
                  credit sales while maintaining same level of risk
              }
        \item
              \pl {Współpraca z departamentami marketingu i sprzedaży w celu optymalizacji procesu sprzedaży
                  kredytów pod względem sprzedażowym i z punktu widzenia ryzyka
              }
              \en {Cooperated with sales and marketing departments to optimize
                  credit sales process to maximize sales while taking risk measures into account
              }
        \item
              \pl{Budowa modelu regresji kwantylowej prognozującego dochody klientów oświadczeniowych (wykrywanie odstających oświadczeń)}
              \en{Building a quantile regression models of customers' stated income (detection of potentially fraudulent statements)}
    \end{itemize}
    \textcolor{accent}{Python, GIT, SAS, T-SQL}

    \divider

    \cvevent{
        \pl{Młodszy specjalista}
        \en{Junior credit risk specialist}
    }{| Bank Pocztowy S.A. |
        \pl{Departament Ryzyka Kredytowego}
        \en{Credit Risk Department}
    }{01/2017 -- 01/2018}{}
    \begin{itemize}
        \item
              \pl{Zbudowanie procesu zasilającego departamentowe repozytoria danych aplikacyjnych z procesu kredytowego}
              \en{Created data pipeline collecting credit application data into a form suitable for further analyses and modelling by the whole department}
        \item
              \pl{Miesięczne raportowanie zarządcze nowego portfela kredytowego}
              \en{Monthly management reporting of credit application data}
        \item \pl{Budowa narzędzi wspierających modelowanie ryzyka kredytowego klientów}
              \en{Created tools to support modelling of credit risk}
        \item
              \pl{Stworzenie algorytmu optymalizującego siatki cenowe kredytów (risk-based pricing)}
              \en{Created algorithm which performed risk based optimization of credits' prices}
    \end{itemize}
    \textcolor{accent}{SAS, PL/SQL}

    \divider

    \cvevent{
        \pl{Administrator baz danych}
        \en{Database administrator} }{| PZU S.A.}{12/2015 -- 12/2016}{}
    \begin{itemize}
        \item
              \pl{Utrzymanie hurtowni danych, budowa narzędzi automatyzujących pracę deweloperów}
              \en{Daily maintenance of DWH, DWH processes, development of utility tools for developers and release automation}
        \item
              \pl{Przygotowywanie wdrożeń, środowisk deweloperskich, instalacja oprogramowania}
              \en{Preparation of releases, development environments, technical support for developers}
    \end{itemize}
    \textcolor{accent}{SAS, PL/SQL}

    % ----- EDUCATION -----
    \newpage
    \cvsection{
        \pl{Wykształcenie}
        \en{Education}
    }
    \cvevent{
        \pl{Analiza danych big data}
        \en{Big data analysis} }{|
        \pl{Szkoła Główna Handlowa}
        \en{Warsaw school of economics}}{09/2017 -- 01/2019}{
        \pl{Warszawa}
        \en{Warsaw}}
    \begin{itemize}
        \item \pl{Przerwane na ostatnim semestrze} \en{Interrupted on last semester}
    \end{itemize}
    \divider
    % \newpage
    \cvevent{
        \pl{Metody ilościowe w ekonomii i systemy informacyjne}
        \en{Quantitative methods in economics and information systems} }{|
        \pl{Szkoła Główna Handlowa}
        \en{Warsaw school of economics}}{09/2017 -- 01/2019}{
        \pl{Warszawa}
        \en{Warsaw}}
    \begin{itemize}
        \item
              \pl{Specjalizacja - Metody analizy decyzji}
              \en{Specialization - Decision analysis methods}
        \item
              \pl{Praca licencjacka - "Porównanie skuteczności modeli CAPM i Famy-Frencha do oceny ryzyka inwestycyjnego na polskiej giełdzie"}
              \en{Bachelor thesis - "Comparison of CAPM and Fama-French investment risk models performance on polish stock market"}
        \item
              \pl{Prowadzenie warsztatów dla studentów w ramach koła naukowego informatyki z SQL, SAS 4GL, Excel}
              \en{Conducted workshops for students on SQL, SAS 4GL, Excel }
    \end{itemize}
    % ----- EDUCATION -----

    % ----- PROJECTS -----
    \cvsection{
        \pl{Projekty}
        \en{Projects}}
    \cvevent{
        \pl{Analiza warszawskiego rynku nieruchomości}
        \en{Polish real estate market analysis } }{\cvrepo{| \faGithub}{https://github.com/eredzik/flats_scraper}}{06/2020 -- }{}
    \begin{itemize}
        \item
              \pl{Web scraping portali mieszkaniowych, zapisanie zebranych danych w bazie danych, deployment na AWS }
              \en{Web scraping of real estate advertisement websites, storage in Postgres, deployment on AWS }
    \end{itemize}
    \cvevent{
        \pl{Aplikacja opensource dla rozliczania zobowiązań podatkowych dla JDG}
        \en{Opensource app for small enterprises for basic taxes calculation} }{\cvrepo{| \faGithub}{https://github.com/eredzik/vatcalc}}{03/2021 -- }{}
    \begin{itemize}
        \item Frontend: Elm, Backend: Python, GraphQL, Fastapi
    \end{itemize}
    % \divider

    \cvsection{
        \pl{Umiejętności}
        \en{Skills}}

    \cvlang{\pl{Modelowanie i analiza danych}\en{Modelling and data analysis}}{
        \pl{Zaawansowany}
        \en{Advanced}}\\
    \begin{itemize}
        \item
              \pl{Modelowanie nadzorowane i nienadzorowane}
              \en{Supervised and unsupervised modelling}
        \item
              \pl{Testowanie hipotez statystycznych}
              \en{Statistical hypothesis testing}
        \item
              \pl{Metodologia bayesowska (budowa modelu, walidacja zbieżności, testowanie hipotez, predykcja)}
              \en{Bayesian methodology (model specification, convergence validation, hypothesis testing, prediction)}
        \item
              \pl{Modele machine learningowe - sieci neuronowe(wykorzystanie pytorch), xgboost}
              \en{Machine learning methods - neural networks (pytorch), xgboost}
        \item
              \pl{Znajomość metodologii IRB do budowy modeli ryzyka kredytowego}
              \en{Knowledge of IRB metodology in context of credit risk modelling}
    \end{itemize}

    % \divider
    \cvlang{Python}{
        \pl{Zaawansowany}
        \en{Advanced}}\\
    \begin{itemize}
        \item
              \pl{Analiza danych: numpy, matplotlib, pymc3, statsmodels, sklearn, pytorch}
              \en{Data analysis: numpy, matplotlib, pymc3, statsmodels, sklearn, pytorch}
        \item
              \pl{Inżynieria danych/backend: sqlalchemy, pyspark, pandas, fastapi, graphql(graphene), dash/plotly}
              \en{Data engineering/backend: sqlalchemy, pyspark, pandas, fastapi, graphql(graphene), dash/plotly}
    \end{itemize}
    % \divider
    \cvlang{SAS}{
        \pl{Zaawansowany}
        \en{Advanced}}\\
    \begin{itemize}
        \item
              \pl{BASE, STAT, SQL, Makroprogramowanie, DI Studio, Management Console, Risk Dimensions}
              \en{BASE, STAT, SQL, Macroprogramming, DI Studio, Management Console, Risk Dimensions}
    \end{itemize}
    % \divider
    \cvlang{
        \pl{Inne}
        \en{Other}}{}\\
    \begin{itemize}
        \item
              \pl{Elm, GIT, SQL, LaTeX, Docker, AWS, bash, GraphQL}
              \en{Elm, GIT, SQL, LaTeX, Docker, AWS, bash, GraphQL}
    \end{itemize}
    % ----- LANGUAGES -----

    \cvsection{
        \pl{Języki}
        \en{Languages}}

    \en{\cvlang{Polish}{Native}}
    \pl{\cvlang{Angielski}{Zaawansowany / C1}}
    \en{\cvlang{English}{Advanced / C1}}

    \pl{\cvlang{Hiszpański}{Komunikatywny / B1}}
    \en{\cvlang{Spanish}{Communicative / B1}}

    \pl{\cvlang{Niemiecki}{Podstawowy / A2}}
    \en{\cvlang{German}{Basic / A2}}


\end{paracol}
\center\vspace*{\fill}
\tiny
\pl{Wyrażam zgodę na przetwarzanie moich danych osobowych dla potrzeb niezbędnych do realizacji procesu rekrutacji (zgodnie z ustawą z dnia 10 maja 2018 roku o ochronie danych osobowych (Dz. Ustaw z 2018, poz. 1000) oraz zgodnie z Rozporządzeniem Parlamentu Europejskiego i Rady (UE) 2016/679 z dnia 27 kwietnia 2016 r. w sprawie ochrony osób fizycznych w związku z przetwarzaniem danych osobowych i w sprawie swobodnego przepływu takich danych oraz uchylenia dyrektywy 95/46/WE (RODO).}
\en{I agree to the processing of personal data provided in this document for realising the recruitment process pursuant to the Personal Data Protection Act of 10 May 2018 (Journal of Laws 2018, item 1000) and in agreement with Regulation (EU) 2016/679 of the European Parliament and of the Council of 27 April 2016 on the protection of natural persons with regard to the processing of personal data and on the free movement of such data, and repealing Directive 95/46/EC (General Data Protection Regulation).}
\end{document}