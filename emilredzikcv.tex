%% If you need to pass whatever options to xcolor
\PassOptionsToPackage{dvipsnames}{xcolor}

%% If you are using \orcid or academicons
%% icons, make sure you have the academicons
%% option here, and compile with XeLaTeX
%% or LuaLaTeX.
% \documentclass[10pt,a4paper,academicons]{altacv}

%% Use the "normalphoto" option if you want a normal photo instead of cropped to a circle
% \documentclass[10pt,a4paper,normalphoto]{altacv}

\documentclass[10pt,a4paper,ragged2e,withhyper]{altacv}
 
%% AltaCV uses the fontawesome5 and academicons fonts
%% and packages.
%% See http://texdoc.net/pkg/fontawesome5 and http://texdoc.net/pkg/academicons for full list of symbols. You MUST compile with XeLaTeX or LuaLaTeX if you want to use academicons.

% Change the page layout if you need to
\geometry{left=1.2cm,right=1.2cm,top=1cm,bottom=1cm,columnsep=0.75cm}

% The paracol package lets you typeset columns of text in parallel
\usepackage{paracol}

% Change the font if you want to, depending on whether
% you're using pdflatex or xelatex/lualatex
\ifxetexorluatex
  % If using xelatex or lualatex:
  \setmainfont{Roboto Slab}
  \setsansfont{Lato}
  \renewcommand{\familydefault}{\sfdefault}
\else
  % If using pdflatex:
  \usepackage[rm]{roboto}
  \usepackage[defaultsans]{lato}
  % \usepackage{sourcesanspro}
  \renewcommand{\familydefault}{\sfdefault}
\fi

% ----- LIGHT MODE -----
\definecolor{SlateGrey}{HTML}{2E2E2E}
\definecolor{LightGrey}{HTML}{666666}
\definecolor{PrimaryColor}{HTML}{001F5A}
\definecolor{SecondaryColor}{HTML}{0039AC}
\definecolor{ThirdColor}{HTML}{F3890B}
\definecolor{BackgroundColor}{HTML}{E2E2E2}
\colorlet{name}{PrimaryColor}
\colorlet{tagline}{PrimaryColor}
\colorlet{heading}{PrimaryColor}
\colorlet{headingrule}{ThirdColor}
\colorlet{subheading}{SecondaryColor}
\colorlet{accent}{SecondaryColor}
\colorlet{emphasis}{SlateGrey}
\colorlet{body}{LightGrey}
\pagecolor{BackgroundColor}   
% ----- DARK MODE -----
%\definecolor{BackgroundColor}{HTML}{242424}
%\definecolor{SlateGrey}{HTML}{6F6F6F}
%\definecolor{LightGrey}{HTML}{ABABAB}
%\definecolor{PrimaryColor}{HTML}{3F7FFF}
%\colorlet{name}{PrimaryColor}
%\colorlet{tagline}{PrimaryColor}
%\colorlet{heading}{PrimaryColor}
%\colorlet{headingrule}{PrimaryColor}
%\colorlet{subheading}{PrimaryColor}
%\colorlet{accent}{PrimaryColor}
%\colorlet{emphasis}{LightGrey}
%\colorlet{body}{LightGrey}
%\pagecolor{BackgroundColor}

% Change some fonts, if necessary
\renewcommand{\namefont}{\Huge\rmfamily\bfseries}
\renewcommand{\personalinfofont}{\small\bfseries}
\renewcommand{\cvsectionfont}{\LARGE\rmfamily\bfseries}
\renewcommand{\cvsubsectionfont}{\large\bfseries}

% Change the bullets for itemize and rating marker
% for \cvskill if you want to
\renewcommand{\itemmarker}{{\small\textbullet}}
\renewcommand{\ratingmarker}{\faCircle}

%% sample.bib contains your publications
%% \addbibresource{sample.bib}

\begin{document}
\name{Emil Redzik}
\tagline{}
%% You can add multiple photos on the left or right
\photoL{4cm}{emil-redzik}

\personalinfo{
    \email{emilredzik1994@email.com}\smallskip
    \phone{+48-500-353-785}
    \location{Warszawa, Polska}\\
    \linkedin{emil-redzik}
}

\makecvheader
%% Depending on your tastes, you may want to make fonts of itemize environments slightly smaller
% \AtBeginEnvironment{itemize}{\small}


%% Set the left/right column width ratio to 6:4.
\columnratio{0}

\begin{paracol}{1}

    % ----- ABOUT ME -----
    \cvsection{About Me}
    \begin{quote}
        Data scientist i inżynier danych. Fascynat bayesowskich metod modelowania danych.
    \end{quote}
    % ----- ABOUT ME -----

    % ----- EXPERIENCE -----
    \cvsection{Doświadczenie}
    \cvevent{Kontraktor }{| mBank | Departament Hurtowni Danych}{06/2019 -- }{}
    Rozwój i utrzymanie systemu ALM służącego do wyliczeń ryzyka płynności.\break

    \begin{itemize}
        \item Masowa refaktoryzacja kodu skutkująca skróceniem czasu dziennych przetwarzań o ok 50\% (8h -> 4h)
        \item Budowa modułu liczącego wskaźnik LCR
        \item Przygotowanie POC przeniesienia systemu na technologię Spark
    \end{itemize}
    \divider

    \cvevent{Specjalista }{| PKO S.A. | Departament Ryzyka Kredytowego}{02/2018 -- 05/2019}{}
    \begin{itemize}
        \item Przebudowa procesu liczenia bazy klientów pre-approved skutkująca zwiększeniem sprzedaży kredytów przy jednoczesnym braku podwyższenia poziomu ryzyka tej populacji
        \item Modelowanie dochodów klientów
    \end{itemize}
    \divider

    \cvevent{Młodszy specjalista }{| Bank Pocztowy S.A. | Departament Ryzyka Kredytowego}{01/2017 -- 01/2018}{}
    \begin{itemize}
        \item Zbudowanie procesu zasilającego departamentowe repozytoria danych aplikacyjnych z procesu kredytowego
        \item Miesięczne raportowanie zarządcze nowego portfela kredytowego
        \item Stworzenie algorytmu optymalizującego siatki cenowe kredytów (risk-based pricing)
    \end{itemize}
    \divider

    \cvevent{Administrator baz danych }{| PZU S.A.}{12/2015 -- 12/2016}{}
    \begin{itemize}
        \item Utrzymanie hurtowni danych
        \item Budowa narzędzi automatyzujących pracę deweloperów
        \item Przygotowywanie wdrożeń, środowisk deweloperskich, instalacja oprogramowania
    \end{itemize}

    % ----- EDUCATION -----
    \cvsection{Wykształcenie}
    \cvevent{Analiza danych big data }{| Szkoła Główna Handlowa}{09/2017 -- 01/2019}{Warszawa}
    \begin{itemize}
        \item Przerwane
    \end{itemize}
    \divider
    \newpage
    \cvevent{Metody ilościowe w ekonomii i systemy informacyjne }{| Szkoła Główna Handlowa}{09/2017 -- 01/2019}{Warszawa}
    \begin{itemize}
        \item Specjalizacja - Metody analizy decyzji
        \item Praca licencjacka - "Porównanie skuteczności modeli CAPM i Famy-Frencha do oceny ryzyka inwestycyjnego na polskiej giełdzie"
        \item Prowadzenie warsztatów dla studentów w ramach koła naukowego informatyki z SQL, SAS 4GL, Excel
    \end{itemize}
    % ----- EDUCATION -----

    % ----- PROJECTS -----
    \cvsection{Projekty}
    \cvevent{Analiza warszawskiego rynku nieruchomości }{\cvrepo{| \faGithub}{https://github.com/eredzik/flats_scraper}}{06/2020 -- }{}
    \begin{itemize}
        \item Web scraping portali mieszkaniowych
        \item Deployment za pomocą dockera
        \item Zapisywanie zebranych danych w bazie danych
    \end{itemize}
    % \divider

    \cvsection{Umiejętności}

    \cvlang{Modelowanie statystyczne}{Zaawansowany}\\
    \begin{itemize}
        \item Modelowanie nadzorowane (modele liniowe, nieliniowe)
        \item Modelowanie nienadzorowane (kmeans, dbscan, GMM, klasteryzacja hierarchiczna)
        \item Testy statystyczne
        \item Metodologia bayesowska (budowa modelu, walidacja zbieżności, testowanie hipotez, predykcja - wykorzystanie pymc3)
        \item Modele machine learningowe - sieci neuronowe(wykorzystanie pytorch), xgboost
        \item Znajomość metodologii IRB do budowy modeli ryzyka kredytowego
    \end{itemize}

    \divider
    \cvlang{Python}{Średniozaawansowany}\\
    \begin{itemize}
        \item Analiza danych: pandas, numpy, matplotlib, pymc3, pyspark, statsmodels, sklearn, pytorch
        \item Webscraping: requests, beautifulsoup, lxml, xpath
        \item Podstawy budowy stron internetowych/dashboardów webowych: flask, dash/bokeh
    \end{itemize}
    \divider
    \cvlang{SAS}{Zaawansowany}\\
    \quad BASE, STAT, SQL, Makroprogramowanie, DI Studio, Management Console, Risk Dimensions

    \divider
    \cvlang{MS SQL}{Średniozaawansowany}\\
    \quad SQL, kursory
    % ----- LANGUAGES -----

    \cvsection{Języki}
    \cvlang{Angielski}{Zaawansowany / C1}
    \divider

    \cvlang{Hiszpański}{Komunikatywny / B1}
    \divider

    \cvlang{Niemiecki}{Podstawowy / A2}


\end{paracol}
\center\vspace*{\fill}
\tiny
Wyrażam zgodę na przetwarzanie moich danych osobowych dla potrzeb niezbędnych do realizacji procesu rekrutacji (zgodnie z ustawą z dnia 10 maja 2018 roku o ochronie danych osobowych (Dz. Ustaw z 2018, poz. 1000) oraz zgodnie z Rozporządzeniem Parlamentu Europejskiego i Rady (UE) 2016/679 z dnia 27 kwietnia 2016 r. w sprawie ochrony osób fizycznych w związku z przetwarzaniem danych osobowych i w sprawie swobodnego przepływu takich danych oraz uchylenia dyrektywy 95/46/WE (RODO).
\end{document}